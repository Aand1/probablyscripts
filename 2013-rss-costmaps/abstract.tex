Costmaps are increasingly being used for purposes other than marking a location in a robot's environment as occupied. Researchers in Human Robot Interaction have used intermediate costmap values to help improve robot navigation in proximity to people. Costmaps with non-zero non-lethal values allow it to represent policies with soft constraints, such as not getting too close to people, or only entering a region when absolutely necessary. This can help improve the quality of interactions and the efficiency of both parties' navigation. 

While constant and Gaussian distributed additions to costmaps are already in use, there has been no general strategy for how to set up the parameters of the additions to ensure the desired behavior is achieved. In this paper, we describe how the parameters for the soft constraints can affect the robot's planned paths, and what constraints on the parameters can be introduced in order to achieve certain behaviors. 

Gaussian distributions, which are oft used to model people's personal space, have several interesting properties. In order to plan a path that is a set distance away from the obstacle, the ratio of the constant added for each cell traversed in a path to the amplitude of the Gaussian needs to be a positive number less than approximately $.57$.  There are also irregularities in the Gaussian parameter space which result in discontinuities in the resulting behavior space. Our mathematic models were check using two tests. The first actually runs a path planning algorithm on simulated costmaps. The second uses the ROS Navigation packages to plan real paths using live data and more advanced planning algorithms. Both tests confirm the properties of the soft constraints that result in the different behaviors. 

%Satisfying soft constraints like these have the capability to improve a person's comfort and safety around a robot, as well as improve the efficiency of the robot. 
