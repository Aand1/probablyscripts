\section{Related Work}
Avoiding a person while navigating in a hallway was one of the original formulations of the problem. Early works like \citet{yoda1997} took a single person into consideration, but assumed a static behavior. Later work by \citet{christensen2004} realized that a more complex world model was needed for such a task and integrated the distance to multiple people and conversation formations into the navigation algorithm. 
\citet{christensen2005} further explored robots navigating around people with explicit consideration of the proxemic zones found in the work of \citet{hall1969}. However, all of these approaches controlled navigation directly with a velocity controller, limiting long term planning and ways the constraints could be formulated. 

A more flexible approach is to use costmaps, where the environment is discretized into a finite number of locations and each location is given a cost. Such grid-based representations were originally formulated by \citet{matthies1988}, with each cell in their Occupancy Grids associated with a random variable that was either occupied or empty. Occupancy Grids have proved very useful over the years in that their volumetric representation allows for unstructured data from heterogeneous sources. These binary values are used to construct a graph with which planning can be done using Dijkstra's algorithm or any of its variants. However, such Occupancy Grids cannot represent non-lethal obstacles. The more advanced representation of costmaps can take any arbitrary number of values, however, in practice, such grids often devolve into the highest values representing occupied, the lowest representing unoccupied, and everything in between is largely untouched. 

Representing a person's personal space has been the primary place where non-lethal, non-neutral costs have been used in costmaps. \citet{dautenhahn:serveseated} %2006 sisbot 
explored the relationship between a robot and the person it is navigating near in a set of live HRI trials, upon which they constructed their recommendations for planning motions with which humans would be comfortable. These included a safety criterion which takes proxemics and the task at hand into consideration, a visibility criterion ensuring the robot approaches in plain view of the human, and a hidden zones criterion which took obstructing obstacles into consideration. These were later formulated into a costmap based system by \citet{sisbot2007}, creating a ``human aware motion planner.'' Each of the added constraints was formulated using a Gaussian distribution based on the person's and robot's locations. The resulting costmap was fed into an A* planner to find the shortest path. 

\citet{kirby:companion} observed that ``Human social conventions are tendencies, rather than strict rules,'' essentially arguing that the binary costmap (occupied/unoccupied) was inadequate for representing human constraints. In addition to minimizing distance and avoiding obstacles, the COMPANION algorithm modeled proxemics and behaviors like passing on the right into the non-lethal values of the costmap. Their work also integrated representations not available in a strict-costmap representation, such as maintaining a default velocity and inertia. 

Non-lethal costmap values have also been used in autonomous vehicles. \citet{likhachev:costmaps} used large constant valued areas to favor driving on the right side of the road and to avoid curves. 

Work by \citet{svenstrup2009} created even more complicated models of personal space, integrating a mixture of four different constraints (Attractor, Rear, Parallel and Perpendicular) modeled as Gaussians. These functions were used to interact with a single person and were scaled based on the person's interest in interacting. Later, \citet{svenstrup2010} expanded this work to maneuver among a field of multiple people while moving toward a goal. 

Among the latest work in this field, \citet{sisbot2011} have expanded their original model to a three dimensional costmap in order to control positioning during hand off tasks, taking safety, visibility and the human's arm comfort into consideration. \citet{fraichard:anthronav} have also created a complex model that, in addition to modeling personal space and visibility, models a ``motion space'' for where a person is likely to move soon. The final value in their costmap is a maximum of the Gaussians created for each of the constraints, with additional constant value areas added for ``interaction areas'' such as the area between a person and a television. 

By and large, most of the additional constraints added to the costmaps follow Gaussian distributions or constant values (with the minor exception of the representation of the intimate personal space in the work by \citet{fraichard:anthronav}, who use Gaussians/constants for the rest of their distributions). 

% christensen Parameters for interactions with people - robot speed, signaling distance, lateral distance






